\subsection{Gram\'atica}
\label{sec:gramatica}
\paragraph{} La siguiente gram\'atica define el MicroHTML:

\[
    G = <N, \Sigma, P, S>
\]

Donde:
\begin{gather*}
    N = \{S, H, HEAD, BODY, HE, SCS, SC, TSC, B\}\\
    \Sigma = \{texto, <HTML>, </HTML>, <HEAD>, </HEAD>, \\
               <TITLE>, </TITLE>, <SCRIPT>, </SCRIPT>, \\
               <BODY>, </BODY>, <H1>, </H1>, <DIV>, </DIV>,\\
               <P>, </P>, <BR>\}\\
    S = S
\end{gather*}

\paragraph{} Que de acuerdo a la clasificaci\'on de Chomsky es una gram\'atica \emph{tipo 0}, o \emph{sin restricciones}. Las producciones ($P$) se detallan en la siguiente subsecci\'on.

\subsection{Producciones - $P$}
\begin{lstlisting}
S       -> <HTML> H </HTML>
 
H       -> HEAD BODY | HEAD | BODY | λ   

HEAD    -> <HEAD> HE </HEAD>

BODY    -> <BODY> B </BODY>

HE      -> SCS <TITLE> texto </TITLE> SCS 

SCS     -> SC SCS | λ

SC      -> <SCRIPT> TSC </SCRIPT>

TSC ->  <HTML> TSC | </HTML> TSC | 
        <HEAD> TSC | </HEAD> TSC |
        <BODY> TSC | </BODY> TSC |
        <TITLE> TSC | </TITLE> TSC |
        <DIV> TSC | </DIV> TSC |
        <H1> TSC | </H1> TSC |
        <P> TSC | </P> TSC |
        <SCRIPT> TSC | <BR> TSC | texto TSC | λ

B   ->  texto B |
        <DIV> B </DIV> B |
        <H1> B </H1> B |
        <P> B </P> B |
        <BR> B |
        λ

\end{lstlisting}